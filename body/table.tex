\section{排版表格}
\LaTeX 中生成简单的表格还是比较方便的,可以用tabular 环境来实现。下面就来做一个论文中经常用到的三线表,如表~\ref{table_1}~。

\begin{table}[htbp!]

\caption{本模板中所用的宏包及功能}\label{table_1}

\begin{tabular}{ccccccccc}
\whline 
宏包名称 & amsmath&caption2&geometry&ulem&xcolor&setspace&hyperref&titletoc \\ 
\hline 
作用 & 数学公式&定制标题&页面设置&下划线&颜色&行距&超链接&目录样式 \\ 
\whline 
\end{tabular}
\end{table}

其实现代码如下:
{\color{green!50!black}
\begin{verbatim}
\begin{table}[htbp!]
\caption{本模板中所用的宏包及功能}\label{table_1}
\begin{tabular}{ccccccccc}
\whline 
宏包名称 & amsmath&caption2&geometry&ulem&xcolor&setspace&hyperref&titletoc \\ 
\hline 
作用 & 数学公式&定制标题&页面设置&下划线&颜色&行距&超链接&目录样式 \\ 
\whline 
\end{tabular}
\end{table}
\end{verbatim}
}
如果表格比较长,那就要用到跨页表格排版宏包longtable了。基本的表格排版情况就介绍这么多,大家感兴趣自己慢慢去探索吧。